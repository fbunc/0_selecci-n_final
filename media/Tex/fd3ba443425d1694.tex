\documentclass[preview]{standalone}

\usepackage[english]{babel}
\usepackage{amsmath}
\usepackage{amssymb}

\begin{document}

\begin{center}
Ésta es una historia de ficción que tiene el objetivo de ayudar a ilustrar algunas intuiciones matemáticas que considero interesantes. 

$$ $$
La intención es que el contenido sea agradable para cualquier persona que no esté familiarizada con los temas tratados.  

$$ $$
También se agregarán detalles que pueden ser de interés para estudiantes que por primera vez se enfrentan a ideas como La Serie de Fourier y les entregan unas fórmulas mágicas sin demasiado contexto. Éste es el contexto que a mi me hubiese gustado que me dieran cuando estudié estos temas. 

$$ $$
Puede ser de interés para docentes, ya que en la descripción hay un link a el código fuente usado para la realización de éste contenido usando SymPy, Manim, NumPy, Matplotlib, y varias librerías de Python adecuadas tanto para el cálculo y la manipulación simbólica como también para la creación de contenido audiovisual en un workflow simplificado que me ha sido de utilidad ya que  soy nuevo Python,ya que vengo de la época en que se usaba Matlab en la universidad.
\end{center}

\end{document}
