\documentclass[preview]{standalone}

\usepackage[english]{babel}
\usepackage{amsmath}
\usepackage{amssymb}

\begin{document}

\begin{center}
Para situarnos en éste experimento mental imaginaremos inicialmente un modelo simplificado de 3 cuerpos celestes tipo Estrella-Planeta-Satélite .(Sol-Tierra-Luna) 

Plantearemos los temas imaginando como serían presentados típicamente en un planeta ficticio (similar al nuestro)  que llamaremos Tau. 

En Tau los equivalentes a nuestros Ptolomeo / Copérnico /Kepler (En la historia de Tau tendremos uno solo que llamaremos simplemente Claudius) y el equivalente a Fourier (que llamaremos Joseph) fueron contemporáneos y amigos ambos estudiantes en la clase del profesor Hard Leon  (Cómo se llamaba en Tau el equivalente del gigante Leonhard Euler). 

En nuestro planeta Tierra, éstos personajes históricos junto con otros que mencionaremos con sus seudónimos  tales como Lissajous (En Tau llamado Antoine ) o J.S. Bach (en Tau lo llamaremos Sebastian), no fueron  contemporáneos (al menos no de la manera que ilustraremos en nuestro planeta ficticio). 

Por eso en nuestra historia paralela de Tau los pensaremos a-temporalmente sólo unidos por las relaciones entre sus descubrimientos individuales  y veremos cómo gracias a su gran profesor Hard Leon todos terminan descubriendo que están expresando lo mismo en diferentes lenguajes, mirando  y modelando la misma realidad fundamental desde distintas perspectivas.
\end{center}

\end{document}
