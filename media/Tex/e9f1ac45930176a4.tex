\documentclass[preview]{standalone}

\usepackage[english]{babel}
\usepackage{amsmath}
\usepackage{amssymb}

\begin{document}

\begin{center}
$$e^{z}=e^{\alpha+i\theta}=e^{\alpha}\cdot e^{i\theta}$$

Si $z=i \theta$ al evaluar la serie de potencias vemos que equivale a sucesivas rotaciones de $90º$  que nos dejan con la verdadera máquina detrás de una expresión tan sencilla, que es: 

$$e^{i\theta} = \sum_{n=0}^{\infty} \left[ \left(-1\right)^n \frac{\theta^{2n}}{(2n)!} \right] + i \sum_{n=0}^{\infty} \left[ \left(-1\right)^n \frac{\theta^{2n+1}}{(2n+1)!} \right]$$


Notar como al evaluar la expansión en serie de potencias siempre obtenemos un ciclo mod 4  antihorario del tipo:

$$1,i,-1,-i $$


Lo que en Tau se suele escribir:

$$\Theta^{\tau},\Theta^{\frac{\tau}{4}},\Theta^{\frac{\tau}{2}},\Theta^{\frac{3\tau}{4}}$$



$$\Theta^{2\pi},\Theta^{\frac{\pi}{2}},\Theta^{\pi},\Theta^{\frac{3\pi}{2}}$$
\end{center}

\end{document}
