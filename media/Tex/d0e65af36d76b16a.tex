\documentclass[preview]{standalone}

\usepackage[english]{babel}
\usepackage{amsmath}
\usepackage{amssymb}

\begin{document}

\begin{center}
Recordando que el coseno en si mismo es el "promedio" de dos acciones fundamentales :

$$a \cos(t)=a\frac{e^{it}+e^{-it}}{2}$$


Y que el seno es la "diferencia" entre dos acciones fundamentales y luego desplazada 90º

$$b \sin(t)=b\frac{e^{it}-e^{-it}}{2i}$$


Es fácil ver que podemos expresar la curva  de forma paramétrica:

$$z(t)=a\cos{t}+i b \sin{t}$$ 

Con 

$$a=R_{_+}+R_{_-}$$ 
$$b=R_{_+}-R_{_-}$$
\end{center}

\end{document}
