\documentclass[preview]{standalone}

\usepackage[english]{babel}
\usepackage{amsmath}
\usepackage{amssymb}

\begin{document}

\begin{center}
Éste nombre era apropiado, ya que  se sabía en la historia de Tau que Claudius usaba mucho el compás como instrumento de dibujo y medición y también usaba números bidimensionales en coordenadas polares y/o cartesianas que le indicaban en qué posición del papel representar cada instante/lugar de su modelo,se sabía que los usaba como notación para representar y aproximar curvas y analizar la mecánica celeste  de forma simplificada. 
$$ $$ 
En castellano, Compás  también es una palabra relacionada con la música mediante la idea de Ritmo  y ésto viene dado,como veremos, por la influencia del trabajo de Sebastian  en Tau.
$$ $$ 
Otro instrumento históricamente relacionado con el nombre elegido para los números mas populares en Tau, es el que usamos para ubicarnos en el espacio según el Campo Magnético de la Tierra que en castellano  llamamos Brújula (en inglés es Compass ).  
$$ $$ 
Y ésto era apropiado en Tau, ya que también tuvieron a un equivalente a J. C. Maxwell que describió la naturaleza ondulatoria del campo electromagnético inspirándose en intuiciones fundamentales de éste estilo. (Como ilustración de ésta relación, pensar cómo  es que el azúcar, al ser una molécula que tiene una determinada quiralidad, modifica el comportamiento de la luz  que pasa a través de un cilindro que contiene un líquido azucarado.(Ver  link en la descripción What is Wiggling ?)
\end{center}

\end{document}
