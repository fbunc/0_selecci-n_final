\documentclass[preview]{standalone}

\usepackage[english]{babel}
\usepackage{amsmath}
\usepackage{amssymb}

\begin{document}

\begin{center}
Ésta es una historia de ficción que tiene el objetivo de ayudar a ilustrar algunas intuiciones matemáticas que considero interesantes. 


$$ $$
Se agregarán detalles que pueden ser de interés para estudiantes que por primera vez se enfrentan a ideas como La Serie de Fourier y les entregan unas fórmulas mágicas sin demasiado contexto. Éste es el contexto que a mi me hubiese gustado que me dieran cuando estudié estos temas. 

$$ $$
En la descripción hay un link a el código fuente usado para la realización de éste contenido usando SymPy, Manim, NumPy, Matplotlib, y varias librerías de Python adecuadas tanto para el cálculo y la manipulación simbólica como también para la creación de contenido audiovisual en un workflow simplificado que me ha sido de utilidad,ya que vengo de la época en que se usaba Matlab en la universidad.

$$ $$ 
Espero que el contenido compartido sea de utilidad para  divulgadores que se están iniciando en la creación de contenido
\end{center}

\end{document}
