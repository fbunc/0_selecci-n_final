\documentclass[preview]{standalone}

\usepackage[english]{babel}
\usepackage{amsmath}
\usepackage{amssymb}

\begin{document}

\begin{center}
\quad\\justifying {First we conceptualize an undirected graph  ${G}$  as a union of a finite number of line segments residing in  ${\quad\\mathbb{C}}$ . By taking our earlier parametrization, we can create an almost trivial extension to  ${\quad\\mathbb{R}}^{3}$ . In the following notation, we write a bicomplex number of a 2-tuple of complex numbers, the latter of which is multiplied by the constant  ${j}$ .  ${z}_0\quad\\in{\quad\\mathbb{C}}_>={0}$  is an arbitrary point in the upper half plane from which the contour integral begins. The function  ${\quad\\tan\quad\\left(\quad\\frac{{{\quad\\theta}-{\quad\\pi}z\quad\\right)}}}:{\quad\\left[{0},{2}{\quad\\pi}\quad\\right)}\quad\\to{\quad\\left[-\quad\\infty,\quad\\infty\quad\\right)}$  ensures that the vertices at  $\quad\\infty$  for the Schwarz-Christoffel transform correspond to points along the branch cut at  ${\quad\\mathbb{R}}_+$ .}
\end{center}

\end{document}
