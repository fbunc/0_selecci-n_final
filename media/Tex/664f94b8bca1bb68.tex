\documentclass[preview]{standalone}

\usepackage[english]{babel}
\usepackage{amsmath}
\usepackage{amssymb}

\begin{document}

\begin{center}
Es importante recordar que cuando  escribimos $e^{z}=e^{lpha}\cdot e^{i	heta}$ estamos abusando de notación para representar una (auto)función,siendo $z  \in \mathbb{C}$ el argumento introducido en $exp(z)$ que está definida según su expansión en series de potencias . 

$$z = lpha + i	heta $$

$$e^{z}=e^{lpha+i	heta}=e^{lpha}\cdot e^{i	heta}$$

$$e^z=exp(z)=\sum_{n=0}^{\infty} rac{z^n}{n!} = 1 + rac{z}{1!} + rac{z^2}{2!} + rac{z^3}{3!} + \cdots$$


$$e^{i	heta}= \sum_{n=0}^{\infty} rac{(i 	heta)^n}{n!} $$

$$e^{i	heta}=\sum_{n=0}^{\infty} \big[ \big(-1
ight)^n rac{	heta^{2n}}{(2n)!} 
ight] + i \sum_{n=0}^{\infty} \big[ \big(-1
ight)^n rac{	heta^{2n+1}}{(2n+1)!} 
ight]$$
\end{center}

\end{document}
