\documentclass[preview]{standalone}

\usepackage[english]{babel}
\usepackage{amsmath}
\usepackage{amssymb}

\begin{document}

\begin{center}
\justifying {First we conceptualize an undirected graph  ${G}$  as a union of a finite number of line segments residing in  ${\mathbb{C}}$ . By taking our earlier parametrization, we can create an almost trivial extension to  ${\mathbb{R}}^{3}$ . In the following notation, we write a bicomplex number of a 2-tuple of complex numbers, the latter of which is multiplied by the constant  ${j}$ .  ${z}_0\in{\mathbb{C}}_>={0}$  is an arbitrary point in the upper half plane from which the contour integral begins. The function  ${\tan\left(\frac{{{\theta}-{\pi}z\right)}}}:{\left[{0},{2}{\pi}\right)}\to{\left[-\infty,\infty\right)}$  ensures that the vertices at  $\infty$  for the Schwarz-Christoffel transform correspond to points along the branch cut at  ${\mathbb{R}}_+$ .}
\end{center}

\end{document}
