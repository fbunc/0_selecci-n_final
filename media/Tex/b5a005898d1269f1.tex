\documentclass[preview]{standalone}

\usepackage[english]{babel}
\usepackage{amsmath}
\usepackage{amssymb}

\begin{document}

\begin{center}
En Tau usan un truco de notación similar al nuestro para las exponenciales complejas, pero basan a todas las
exponenciales complejas de módulo y argumento unitario con un número $\Theta=e^i$ y lo aprovechan para lograr 
expresiones más fáciles de leer. 

$$\Theta^{\theta}=e^{i\theta} = \cos(\theta) + i\sin(\theta)$$
\end{center}

\end{document}
