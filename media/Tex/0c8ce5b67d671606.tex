\documentclass[preview]{standalone}

\usepackage[english]{babel}
\usepackage{amsmath}
\usepackage{amssymb}

\begin{document}

\begin{center}
Es importante recordar que cuando  escribimos $e^{z}=e^{\alpha}\cdot e^{i\theta}$ estamos abusando de notación para representar una (auto)función,siendo $z  \in \mathbb{C}$ el argumento introducido en $exp(z)$ que está definida según su expansión en series de potencias . 

$$z = \alpha + i\theta $$

$$e^{z}=e^{\alpha+i\theta}=e^{\alpha}\cdot e^{i\theta}$$

$$e^z=exp(z)=\sum_{n=0}^{\infty} \frac{z^n}{n!} = 1 + \frac{z}{1!} + \frac{z^2}{2!} + \frac{z^3}{3!} + \cdots$$


$$e^{i\theta}= \sum_{n=0}^{\infty} \frac{(i \theta)^n}{n!} $$

$$e^{i\theta}=\sum_{n=0}^{\infty} \left[ \left(-1\right)^n \frac{\theta^{2n}}{(2n)!} \right] + i \sum_{n=0}^{\infty} \left[ \left(-1\right)^n \frac{\theta^{2n+1}}{(2n+1)!} \right]$$
\end{center}

\end{document}
