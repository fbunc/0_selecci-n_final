\documentclass[preview]{standalone}

\usepackage[english]{babel}
\usepackage{amsmath}
\usepackage{amssymb}

\begin{document}

\begin{center}
Ésta es una historia de ficción que tiene el objetivo de ayudar a ilustrar algunas intuiciones matemáticas que considero interesantes. 

$$ $$
La intención es que el contenido sea agradable para cualquier persona que no esté familiarizado con los temas tratados.  

$$ $$
Pero al mismo tiempo se agregarán detalles que pueden ser de interés para estudiantes de bachillerato o ya en nivel universitario. 

$$ $$
Puede ser de interés para docentes, ya que en la descripción hay un link a el código fuente usado para la realización de éste contenido usando SymPy, Manim, NumPy, Matplotlib, y varias librerías de Python adecuadas tanto para el cálculo y la manipulación simbólica como también para la creación de contenido audiovisual en un workflow simplificado que me ha sido de utilidad ya que  soy nuevo Python y al igual que muchas personas viendo, vengo de la época en que se usaba Matlab.
\end{center}

\end{document}
