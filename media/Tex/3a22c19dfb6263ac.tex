\documentclass[preview]{standalone}

\usepackage[english]{babel}
\usepackage{amsmath}
\usepackage{amssymb}

\begin{document}

\begin{center}
Para situarnos en éste experimento mental imaginaremos inicialmente un modelo simplificado de las órbitas de 3 cuerpos celestes del tipo Estrella-Planeta-Satélite (Sol-Tierra-Luna) 

$$ $$
Imaginaremos cómo serían presentados típicamente éstos temas en un planeta matemático ficticio que llamaremos Tau. 

$$ $$ 

Imaginaremos algunos personajes  de Tau, que llamaremos Claudius,Joseph,Antoine y Sebastian. 

Todos se conocían de la universidad, donde asistían juntos a la cátedra de Hard Leon, considerado el más grande matemático de la historia de Tau.
$$ $$ 
Veremos como Claudius está inspirado en personajes como Ptolomeo-Kepler-Copernico-etc, y cómo su obsesión inicial era sumar epiciclos.  
$$ $$ 
Joseph está inspirado en Fourier, ya que era el que estaba obsesionado en modelar el comportamiento del calor usando ecuaciones diferenciales.  
$$ $$ 
Antoine está inspirado en Lissajous y sus figuras y sus experimentos como el armonógrafo, que en Tau será una especie de osciloscopio SteamPunk.
$$ $$ 
Hard Leon, está inspirado en Leonhard Euler que imaginaremos que era el profesor de Análisis Matemático de los personajes mencionados.

$$ $$
\end{center}

\end{document}
