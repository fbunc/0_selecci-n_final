\documentclass[preview]{standalone}

\usepackage[english]{babel}
\usepackage{amsmath}
\usepackage{amssymb}

\begin{document}

\begin{center}
A lo que nosotros llamamos Números Complejos $\mathbb{C}$, típicamente expresados :

$$z=a+ib=re^{i\theta}$$ 

En Tau éstos eran llamados Números Compás, por su relación con las  circunferencias y los vectores que fácilmente pueden representar.
Notar que en inglés Compass también significa brújula y en castellano también significa ritmo 

$$ $$


Los habitantes de Tau vivían en armonía con la naturaleza y las matemáticas eran parte fundamental de la cultura del planeta. 

Se enseñaban las ideas profundamente, con pasión y arte, tanto para aplicarlas a la creación de tecnologías que avanzan la sociedad,como también para describir el universo  y transmitir de generación en generación los conocimientos e intuiciones fundamentales. 

$$ $$ 

En la descripción del video habrá enlaces a información complementaria, para distintos niveles de familiaridad con los temas tratados.
\end{center}

\end{document}
