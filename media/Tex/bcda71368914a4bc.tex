\documentclass[preview]{standalone}

\usepackage[english]{babel}
\usepackage{amsmath}
\usepackage{amssymb}

\begin{document}

\begin{center}
Para situarnos en éste experimento mental imaginaremos inicialmente un modelo simplificado de las órbitas de 3 cuerpos celestes del tipo Estrella-Planeta-Satélite (Sol-Tierra-Luna) 

$$ $$
Imaginaremos como serían presentados típicamente éstos temas en un planeta ficticio que llamaremos Tau. 

$$ $$ 

Imaginaremos la historia de tres personajes históricos de Tau, que llamaremos Claudius, Joseph , Antoine y Sebastian. 

$$ $$ 

Claudius, por sus descubrimientos primitivos y será equivalente a nuestros Ptolomeo-Copérnico-Kepler-etc (En la historia de Tau tendremos uno solo que llamaremos simplemente Claudius) y el equivalente a Fourier (que llamaremos Joseph) fueron contemporáneos y amigos ambos estudiantes en la clase del profesor Hard Leon  (Cómo se llamaba en Tau el equivalente del gigante Leonhard Euler). 

En nuestro planeta Tierra, éstos personajes históricos junto con otros que mencionaremos con sus seudónimos  tales como Lissajous (En Tau llamado Antoine ) o J.S. Bach (en Tau lo llamaremos Sebastian), no fueron  contemporáneos (al menos no de la manera que ilustraremos en nuestro planeta ficticio). 

Por eso en nuestra historia paralela de Tau los pensaremos a-temporalmente sólo unidos por las relaciones entre sus descubrimientos individuales  y veremos cómo gracias a su gran profesor Hard Leon todos terminan descubriendo que están expresando lo mismo en diferentes lenguajes, mirando  y modelando la misma realidad fundamental desde distintas perspectivas.
\end{center}

\end{document}
