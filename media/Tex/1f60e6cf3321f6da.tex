\documentclass[preview]{standalone}

\usepackage[english]{babel}
\usepackage{amsmath}
\usepackage{amssymb}

\begin{document}

\begin{center}
Para dibujar la historia usaremos Números Complejos, pero justamente por que al usarlos todo resulta mucho mas fácil e intuitivo.

Por eso, no dejes de ver el video si no estás familiarizado con el tema, ya que espero que te quedes con intuiciones que te ayudarán a entenderlo luego. 

La idea original de éste vídeo era hacer una introducción básica a la idea de número bidimensional. Pero si entraba en todas las formalidades básicas en el formato típico  de la educación académica, nunca iba a poder terminar la historia que une todos los hilos. Es por eso que en la descripción habrá enlaces a introducciones formales recomendadas apropiadas para distintos niveles de familiaridad con los temas tratados. 

Imaginemos que en Tau a ésta idea de número de 2 dimensiones, al haber sido su línea histórica de descubrimientos muy distinta a la nuestra los terminaron llamando Números Compás.
\end{center}

\end{document}
